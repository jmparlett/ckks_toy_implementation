
\documentclass{beamer}
\usepackage[utf8]{inputenc}
\usepackage[english]{babel}
\usepackage{amsmath}
\usepackage{amssymb}
\usepackage{fancyhdr}
\usepackage{pgfplots}
\usepackage{setspace}
\usepackage{listings}
\pgfplotsset{compat=1.17} 
\usepackage{enumerate}
\usepackage{algorithm}
\usepackage{algpseudocode}
% \geometry{a4paper} % or letter or a5paper or ... etc
% \geometry{landscape} % rotated page geometry
% \usepackage[margin=2cm]{geometry}
\usepackage{minted}
\usepackage[most]{tcolorbox}
\newtcolorbox{tb}[1][]{%
  sharp corners,
  enhanced,
  colback=white,
  height=6cm,
  attach title to upper,
  #1
}

%These setting will make the code areas look Pretty
\lstset{
	escapechar=~,
	numbers=left, 
	%numberstyle=\tiny, 
	stepnumber=1, 
	firstnumber=1,
	%numbersep=5pt,
	language=C,
	% stringstyle=\itfamily,
	%basicstyle=\footnotesize, 
	showstringspaces=false,
	frame=single,
  upquote=true
}

% created 2023-February-5 %
% Theme choice:
% \usetheme{AnnArbor}
\usetheme{focus}
% Title page details: 
\title{ckks Encoding}
\author{Jonathan Parlett}
\date{\today}

\begin{document}

% Title page frame
\begin{frame}
    \titlepage
\end{frame}


\begin{frame}{What well cover}
	\begin{itemize}[<+->]
		\item Cyclotomic polynomials and their degrees
		\item A simplified encoding scheme from polynomials with complex coefficients to
		vectors with complex coefficients
		\item The actual encoding scheme from polynomials with integer coefficients
		to vectors of complex coefficients. 
	\end{itemize}
\end{frame}

\begin{frame}{Cyclotomic polynomials}
	\begin{itemize}[<+->]
		\item The $n$-th Cyclotomic polynomial is defined as $\Phi_n(x) = \Pi_{1 \le k \le n | \gcd(k,n) = 1} (x - e^{2i\pi\frac{k}{n}})$
		\item From the constraint that $gcd(k,n) = 1$ you may be able to infer that the degree of the $n$-th
		cyclotomic polynomial is equal to $\rho(n)$ where $\rho$ is Eulers totient function.
		\item This property will be important to consider when you we select a cyclotomic to use for our
		encoding
		\item Another important property of cyclotomics is that there roots are complex conjugates
		of each other. To see this lets look at the 8-th cyclotomic $X^4 + 1$
		\item $\Phi_8(x) = (x - e^{2i \pi \frac{1}{8}})(x - e^{2i \pi \frac{3}{8}})(x - e^{2i \pi \frac{5}{8}})(x - e^{2i \pi \frac{7}{8}})$
	\end{itemize}
\end{frame}

\begin{frame}{Cyclotomic polynomial roots are complex conjugates: Example}
	\begin{itemize}[<+->]
		\item $x^4 + 1 = \Phi_8(x) = (x - e^{2i \pi \frac{1}{8}})(x - e^{2i \pi \frac{3}{8}})(x - e^{2i \pi \frac{5}{8}})(x - e^{2i \pi \frac{7}{8}})$
		\item $ = (x - (\frac{\sqrt{2}}{2} + i\frac{\sqrt{2}}{2}))(x - e^{2i \pi \frac{3}{8}})(x - e^{2i \pi \frac{5}{8}})(x - (\frac{\sqrt{2}}{2} + i\frac{\sqrt{2}}{2}))$
		\item $ = (x - (\frac{\sqrt{2}}{2} + i\frac{\sqrt{2}}{2}))(x - (-\frac{\sqrt{2}}{2} + i\frac{\sqrt{2}}{2}))(x - (-\frac{\sqrt{2}}{2} - i\frac{\sqrt{2}}{2}))(x - (\frac{\sqrt{2}}{2} + i\frac{\sqrt{2}}{2}))$
		\item $ = (x -\frac{\sqrt{2}}{2} - i\frac{\sqrt{2}}{2})(x +\frac{\sqrt{2}}{2} - i\frac{\sqrt{2}}{2})(x +\frac{\sqrt{2}}{2} + i\frac{\sqrt{2}}{2})(x -\frac{\sqrt{2}}{2} - i\frac{\sqrt{2}}{2})$
		\item Grouping real and imaginary parts we can see that the 1st root is the complex of the 4th 
		and the 2nd root is the complex conjugate of the 3rd
		\item $= ([x -\frac{\sqrt{2}}{2}] - i\frac{\sqrt{2}}{2})([x +\frac{\sqrt{2}}{2}] - i\frac{\sqrt{2}}{2})([x +\frac{\sqrt{2}}{2}] + i\frac{\sqrt{2}}{2})([x -\frac{\sqrt{2}}{2}] - i\frac{\sqrt{2}}{2})$
		\item This fact will become important when we are discussing the mapping of polynomials with integer
		coefficients to complex vectors
	\end{itemize}
\end{frame}

\begin{frame}{Simplified encoding scheme}
	\begin{itemize}[<+->]
		\item The input messages in the CKKS scheme are vectors of complex numbers $z \in \mathbb{C}^N$
		where $N$ is called the degree modulus for reasons that will become obvious
		\item All homomorphic operations in the CKKS scheme are performed on polynomials in the ring
		$\frac{\mathbb{Z}[X]}{X^N + 1}$. The homomorphic properties of these operations come as a consequence
		of the properties of these polynomial rings
		\item So the first big thing to understand in order to fully understand CKKS is this encoding
		scheme. How do we get from our complex vectors to our plaintext polynomials
		\item The ultimate is goal to be able to fully understand the map that defines CKKS encoding algorithm
		$$\sigma^{-1} : \mathbb{C}^N \to \frac{\mathbb{Z}[X]}{X^N + 1}$$

	\end{itemize}
\end{frame}

\begin{frame}{Simplified encoding scheme}
	\begin{itemize}[<+->]
		\item To start we will first understand the simpler map from complex vectors to polynomials with complex
		coefficients 
		$$\sigma^{-1} : \mathbb{C}^N \to \frac{\mathbb{C}[X]}{X^N + 1}$$
		\item The forward map may be easier to consider
		$$\sigma: \frac{\mathbb{C}[X]}{X^N + 1} \to \mathbb{C}^N$$
		% \item This will probably look familiar from previous lectures as this map is really just
		% the $n$-dimensional DFT matrix and its inverse, but lets go through the process regardless
	\end{itemize}
\end{frame}

\begin{frame}{Simplified encoding scheme}
	\begin{itemize}[<+->]
		\item First lets notice a few things about the modulus $N$. The ring $\frac{\mathbb{Z}[X]}{X^N + 1}$
		is in some way defined by a cyclotomic polynomial of degree $N$.
		\item In CKKS they usually consider $N$ to be a power of 2, $N = 2^k$. So we need cyclotomic
		polynomial of degree $N = 2^k$.
		\item Since as we noted earlier the degree of the $n$-th cyclotomic is equal to $\rho(n)$
		it is easy to find a cyclotomic of the appropriate degree as $\rho(2^{k+1}) = 2^k$, since the
		only numbers that have a common factor with $2^k$ will be only the even numbers less than $2^k$
	\end{itemize}
\end{frame}

\begin{frame}{Simplified encoding scheme}
	\begin{itemize}[<+->]
		\item Once we have our cyclotomic of degree $N$ to map a polynomial of degree $N$ to
		a vector in $C^N$ we simply evaluate that polynomial at the $N$ roots our our cyclotomic
		\item For a single root of unity $\omega$, and a polynomial $P(x)$ we have
		 $$P(\omega) = a_N\omega^N + a_{N-1}\omega^{N-1} + \cdots + a_0\omega^0 = b_i$$
		\item Our output vector $b \in C^N$ will be the vector of $b_i$s for all roots of our cyclotomic
		which is ultimately the result of this matrix vector product for $\Phi_8(x) = X^4 + 1$
		\item 
		\begin{center}
			\[
			\begin{pmatrix}
				1 & (e^i\pi/4)^1 &  (e^i\pi/4)^2 &  (e^i\pi/4)^3 & (e^i\pi/4)^4\\  
				1 & (e^i\pi 3/4)^1 &  (e^i\pi 3/4)^2 &  (e^i\pi 3/4)^3 & (e^i\pi 3/4)^4\\  
				1 & (e^i\pi 5/4)^1 &  (e^i\pi 5/4)^2 &  (e^i\pi 5/4)^3 & (e^i\pi 5/4)^4\\  
				1 & (e^i\pi 7/4)^1 &  (e^i\pi 7/4)^2 &  (e^i\pi 7/4)^3 & (e^i\pi 7/4)^4\\  
			\end{pmatrix} \cdot 
			\begin{pmatrix}
				a_0\\
				a_1\\
				a_2\\
				a_3\\
				a_4\\
			\end{pmatrix} =
			\begin{pmatrix}
				b_0\\
				b_1\\
				b_2\\
				b_3\\
				b_4\\
			\end{pmatrix}
			\]
		\end{center}
	\end{itemize}
\end{frame}
\begin{frame}{Simplified encoding scheme}
	\begin{itemize}[<+->]
	\item 
		\begin{center}
			\[
			\begin{pmatrix}
				1 & (e^i\pi/4)^1 &  (e^i\pi/4)^2 &  (e^i\pi/4)^3 & (e^i\pi/4)^4\\  
				1 & (e^i\pi 3/4)^1 &  (e^i\pi 3/4)^2 &  (e^i\pi 3/4)^3 & (e^i\pi 3/4)^4\\  
				1 & (e^i\pi 5/4)^1 &  (e^i\pi 5/4)^2 &  (e^i\pi 5/4)^3 & (e^i\pi 5/4)^4\\  
				1 & (e^i\pi 7/4)^1 &  (e^i\pi 7/4)^2 &  (e^i\pi 7/4)^3 & (e^i\pi 7/4)^4\\  
			\end{pmatrix} \cdot 
			\begin{pmatrix}
				a_0\\
				a_1\\
				a_2\\
				a_3\\
				a_4\\
			\end{pmatrix} =
			\begin{pmatrix}
				b_0\\
				b_1\\
				b_2\\
				b_3\\
				b_4\\
			\end{pmatrix}
			\]
		\end{center}
		\item Here we can see we have the coefficient vector {\bf a} that uniquely determines the polynomial
		and the output vector {\bf b}. 
		\item Its clear from this equation that given one we can compute the other
		and since this is a square matrix there is one and only one solution
		\item so it should be intuitive that this transformation defines an isomorphism between $\mathbb{C}^4$ and $\frac{\mathbb{C}[X]}{X^4 + 1}$
	\end{itemize}
\end{frame}

\begin{frame}{Simplified encoding scheme}
	\begin{itemize}[<+->]
		\item At this point we have defined our simplified map and its inverse as essentially the equation we showed
		in the previous slide
		$$\sigma^{-1} : \mathbb{C}^N \to \frac{\mathbb{C}[X]}{X^N + 1}$$
		\item The CKKS map/algorithm $\sigma^{-1} : \mathbb{C}^N \to \frac{\mathbb{Z}[X]}{X^N + 1}$ adds further structure
		in order to place restrictions on this map to ensure that we encode our complex vectors as polynomials with
		integer coefficients only


	\end{itemize}
\end{frame}

\end{document}